% !TeX program = lualatex

\documentclass{../awesomecv}

\newcommand{\name}{Nikos Tsiknakis}

\usepackage{geometry}
\geometry{
  left     =  20mm,
  right    =  20mm,
  bottom   =  23mm,
  top      =  10mm,
  footskip =  10mm
}

\usepackage{fancyhdr}
\pagestyle{fancy}
\fancyhf{}
\renewcommand{\headrulewidth}{0pt}
\lfoot{{\footnotesize\color{\secondfont}\today}}
\cfoot{{\footnotesize\color{\secondfont}\name}}
\rfoot{{\footnotesize\color{\secondfont}\thepage}}

\documentcolor{blue}
\secondcolor{base01}
\thirdcolor{base01}
\sectioncolor{base02}

\titlethicknessleft{3mm}

\voiceleftsize{2.1cm}
\voicerightsize{12cm}

\begin{document}

% \pagenumbering{gobble}

\begin{titlebox}
  \authorname{Nikos Tsiknakis}{\emph{Curriculum Vit\ae}}

  \includegraphics[scale=0.1]{images/tsiknakis.png}
  % \qrcode{../pics/qrcode.png}

  \tcblower

  \begin{showinfo}
    \birthdate{$17^{\text{th}}$ April, 1996}
    \location{Heraklion, Greece}
    \phone{+30 6947478728}
    \firstMail{\href{mailto:tsiknakisn@gmail.com}{tsiknakisn@gmail.com}}
    \otherMail{\href{mailto:tsiknakisn@ics.forth.gr}{tsiknakisn@ics.forth.gr}}
    \github{ \href{https://github.com/tsikup}{tsikup} }
    % \twitter{https://twitter.com/BryanCranston}
    \generic{\faLinkedin}{ \href{https://www.linkedin.com/in/tsiknakisn/}{tsiknakisn} }
    \generic{\faFirefox}{ \href{https://tsikup.github.io}{tsikup.github.io} }
  \end{showinfo}

\end{titlebox}

\updated{}
% \vspace{20pt}

% ********************* %
% *     EDUCATION     * %
% ********************* %
\opensection{\faBook}{Education}
\begin{describesection}

  \leftside{\bf Sep 2014 - Jul 2019}
  \rightsidecomplex{BEng - MEng in Electrical and Computer Engineering}{University of Patras}{}{5 years Integrated Masters 300 ECTS degree - Graduated 3rd in my class with a grade of 8.3/10}

  \leftside{Thesis}
  \rightsideplain{\textsc{Stereo Vision of Dual View Meteosat Images} \href{https://tsikup.github.io/assets/pdf/tsik-thesis.pdf}{(\textcolor{blue}{link})}}
  \leftside{Supervisor}
  \rightsideplain{\href{http://www.ece.upatras.gr/skodras/}{Prof. Athanassios Skodras} \vspace{6pt}}

  \leftside{Thesis Details}
  \rightsideplain{This thesis was conducted in collaboration with the Vrije Universiteit Brussel (VUB) and the Royal Meteorological Institute of Brussels (KMI). \vspace{4pt}}

  \leftside{}
  \rightsideplain{Its central objective was the application of advanced stereo vision techniques for the estimation of cloud height based on images from the MSG-3 and MSG-1 satellites. \vspace{4pt}}

  \leftside{}
  \rightsideplain{
    In the context of the thesis I have specifically focused on the following:
    \begin{enumerate}
      \item Extensive review of prior work regarding image registration methods.
      \item Implementation of several state-of-the-art image registration methods. Specifically, I have compared two major algorithmic family of methods, i.e.
      \begin{enumerate}
          \item One operating on the pixel domain, which is based on the maximization of the Mutual Information of the image pair and
          \item One operating on the low-level features of the images. In this approach, the features were extracted using the SIFT and SURF algorithms. The transformation model for registering the images was estimated using these features and the RANSAC algorithm.
      \end{enumerate}
      \item  Review and application of Graph Cut algorithms based on the Markov Random Fields theory to extract the disparity map.
    \end{enumerate}
  }

  \leftside{\bf 2010 - 2014}
  \rightsidecomplex{Secondary Education}{2nd High School}{Heraklion, Greece}{Graduated with hounors and a grade of 19.0/20.0}

  \leftside{}
  \rightsideplain{In 2014 I participated at the National University Entrance Exams in which I obtained a GPA: 18.753/20.000}

\end{describesection}

% ********************* %
% *     EXPERIENCE    * %
% ********************* %
\opensection{\faBlackTie}{Experience}
\begin{describesection}

  % WORK 1
  \leftside{\bf Sep 2019 - }
  \rightsidecomplex{Software Engineer}{Computational Biomedicine Laboratory ICS FORTH}{Heraklion, Greece}{}

  \leftside{}
  \rightsideplain{
    I am employed as a Software Engineer under the supervision of Associate Professor Kostas Marias in the Computational Biomedicine Lab of the Institute of Computer Science, Foundation of Research and Technology Hellas.
  }

  \leftside{}
  \rightsideplain{

  I work in:

  \begin{enumerate}
    \item the H2020 funded project InSilc (\url{https://insilc.eu/}) in which my main responsibilities relate to the implement and testing of a number of algorithmic approaches for the registration of IVUS images before and after stent procedure.
    \item the H2020 funded project SeeFar (\url{https://www.see-far.eu/}) on the development and evaluation of methods for the detection and the progression of retina diseases (diabetic retinopathy and  age-related macular degeneration), as well as diseases that are not related with the eyes health, but signs of the diseases can be detected by the analysis of retina images (e.g. cardiovascular risk). Both traditional (featured based) machine learning approaches as well as deep learning methods will be explored and evaluated  with respect to accuracy, sensitivity and specificity in diagnosis and prediction.
  \end{enumerate}
  }

  % Intership
  \leftside{\bf Jul 2018 - Sep 2018}
  \rightsidecomplex{Intern}{Computational Biomedicine Laboratory ICS FORTH}{Heraklion, Greece}{}

  \leftside{}
  \rightsideplain{
    The main topics I worked on and skills I developed during my internship are:
    \begin{enumerate}[leftmargin=20pt]
      \item Methodological approaches for the execution of systematic literature surveys
      \item Review of stereo vision algorithms and initial implementations of selected such algorithms
      \item Review of Machine/Deep Learning methods and experimentation with the Tensorflow/Keras frameworks
    \end{enumerate}
  }

  % \subdescription{base01}{\faTrophy}{Products}

  % \leftside{name}
  % \rightsideplain{\color{blue}Blue Sky}

  % \leftside{description}
  % \rightsideplain{\emph{Blue Sky is the bomb, bitch. Have you ever seen a
  %     quality of 99.1\%?}}

\end{describesection}

% ********************* %
% *      SEMINARS     * %
% ********************* %
\opensection{\faFlask}{Seminars \& Workshops}
\begin{describesection}

  \leftside{\bf 28-31 Aug 2018}
  \rightsidecomplex{Drone School \& Workshops: Deep learning and Computer vision for drone imaging and cinematography}{Icarus Group CSD AUTH}{}{\url{http://icarus.csd.auth.gr/activities/droneschool/} \vspace{5pt}}

  \leftside{}
  \rightsideplain{I remotely attented the summer school which provided an in depth overview of the various computer vision and deep learning problems encountered in drone imaging and cinematography and corresponding applicable methods. \vspace{15pt}}

  \leftside{\bf 12-14 Feb 2018}
  \rightsidecomplex{MEDDAYS 2018}{Sophia Antipolis}{}{\url{http://leat.unice.fr/MEDDAYS2018/\#page=home} \vspace{5pt}}

  \leftside{}
  \rightsideplain{I was selected to be one of 35 University students in total from Greece, Italy, Spain, Algeria, Morocco, Tunisia, Lebanon, Egypt and Turkey, to attend the fifth edition of the «Mediterranean Days@Campus SophiaTech» event. The goal was to present to selected students, the Sophia Antipolis campus, the research activities of selected highly visible laboratories with a focus on offered international master and doctoral studies. \vspace{15pt}}

  \leftside{\bf 18-20 Oct 2017}
  \rightsidecomplex{Autumn School Aeroworks}{University of Patras}{}{\url{http://www.aeroworks2020.eu/school} \vspace{5pt}}

  \leftside{}
  \rightsideplain{The above summer school focused on providing insights and high tech knowledge on Aerial Robotics and specifically in the following domains: (a) Aerial manipulation, (b) Vision for aerial manipulation, (c) Cooperative aerial coverage, (d) Modeling and Control of UAVs, (e) Estimation and Sensor fusion for UAVs, (f) Aerial reconstruction and inspection. \vspace{15pt}}

  \leftside{\bf 10-15 Jul 2016}
  \rightsidecomplex{1st Interdisciplinary Summer School on Privacy (ISP 2016)}{Nijmegen, Netherlands}{}{\url{https://isp.cs.ru.nl/2016/} \vspace{5pt}}

  \leftside{}
  \rightsideplain{The interdisciplinary summer school on privacy provided an intensive one week academic postgraduate programme teaching privacy from a technical, legal and social perspective. The goal of the summer school was to provide students with a solid background in the theory of privacy construction, modelling and protection from these three different perspectives. \vspace{15pt}}

  \leftside{\bf 22-24 Apr 2016}
  \rightsidecomplex{9 th Panhellenic Conference of ECE Students}{Chania, Greece}{}{\url{https://sfhmmy9.sfhmmy.gr/} \vspace{5pt}}

\end{describesection}

% ********************* %
% *       AWARDS      * %
% ********************* %
\opensection{\faDiamond}{Prizes and Awards}
\begin{describesection}

  \leftside{\bf Feb 2014}
  \rightsidecomplex{Best 1st year's project, Introduction to Programming Course}{ECE University of Patras}{}{}

  \leftside{}
  \rightsideplain{I was part of a team (7 members) that was awarded the best 1st year’s project award for the Introduction to Programming Course. We worked with the programming language Python for the design, implementation and testing of an integrated Image Editor and Viewer supporting an interactive graphical user interface.}

\end{describesection}

% *************************** %
% *       PUBLICATIONS      * %
% *************************** %
\opensection{\faBook}{Publications in International Conferences}
\begin{describesection}

  \leftside{1.}
  \rightsideplain{C. Spanakis, E. Mathioudakis, N. Kampanis, N. Tsiknakis, K. Marias, Renyi divergence and non-deterministic subsampling in Rigid Image Registration, 2019 IEEE International Conference on Imaging Systems and Techniques (IST) (accepted)}

\end{describesection}


% ************************ %
% *     CERTIFICATES     * %
% ************************ %
\opensection{\faCertificate}{Certificates}
\begin{describesection}

  \leftside{\bf Oct 2019}
  \rightsidecomplex{Deep Learning Specialization}{Coursera}{}{\url{https://www.coursera.org/account/accomplishments/specialization/2X3DAMJNRWQ4} \vspace{5pt}}

  \leftside{\bf Oct 2019}
  \rightsidecomplex{Sequence Models}{Coursera}{}{\url{https://www.coursera.org/account/accomplishments/verify/KCLX2DCF9J89} \vspace{5pt}}

  \leftside{\bf Sep 2019}
  \rightsidecomplex{Convolutional Neural Networks}{Coursera}{}{\url{https://www.coursera.org/account/accomplishments/verify/4GRNSJPW5KW9} \vspace{5pt}}

  \leftside{\bf Sep 2019}
  \rightsidecomplex{Structuring Machine Learning Projects}{Coursera}{}{\url{https://www.coursera.org/account/accomplishments/verify/EGYB8W3WLY28} \vspace{5pt}}

  \leftside{\bf Sep 2019}
  \rightsidecomplex{Improving Deep Neural Networks: Hyperparameter tuning, Regularization and Optimization}{Coursera}{}{\url{https://www.coursera.org/account/accomplishments/verify/FPQQ6YWKQK7P} \vspace{5pt}}

  \leftside{\bf Sep 2019}
  \rightsidecomplex{Neural Networks and Deep Learning}{Coursera}{}{\url{https://www.coursera.org/account/accomplishments/verify/P7AE9CDMRBEE} \vspace{5pt}}

\end{describesection}

% ********************* %
% *       SKILLS      * %
% ********************* %
\opensection{\faDiamond}{SKILLS}
\begin{describesection}

  \leftside{\bf Programming Skills}
  \rightsidecomplex{Advanced}{C, Python, Java, Matlab, SQL, HTML, CSS, JavaScript}{}{}

  \leftside{}
  \rightsidecomplex{Very Good}{C++}{}{}

  \leftside{}
  \rightsidecomplex{Good}{Prolog}{}{}

  \leftside{\bf Tools, Frameworks \& Technologies}
  \rightsidecomplex{\vspace{-11pt}}{Tensorflow \& Keras Frameworks \newline
  Django, Node.js, SQL (MySQL), NoSQL (MongoDB) \newline
  Unix/Linux OS and Git Version Control \newline
  Texas Instruments DSK6713 Digital Signal Processor \newline
  Microsoft Office \& LaTeX \newline
  Abode Lightroom \newline
  AutoCAD \newline
  }{}{}

\end{describesection}

% ************************ %
% *     UNI PROJECTS     * %
% ************************ %
\opensection{\faBug}{Major University Projects}
\begin{describesection}

  \leftside{\bf Oct 2017 – Feb 2018}
  \rightsidecomplex{IEEE Signal Processing Society SPCup2018}{5 members group - ECE University of Patras}{}{\url{https://piazza.com/ieee_sps/other/spcup2018/home} \vspace{5pt}}

  \leftside{}
  \rightsideplain{The specific challenge focused on the topic: A forensic camera model identification challenge. The goal of this competition was to build a system capable of determining the type of camera (manufacturer and model) that captured a digital image without relying on metadata. We developed a system based on machine learning techniques that could identify the camera type with an accuracy of 71\%. \vspace{15pt}}

  \leftside{\bf 2014-2019}
  \rightsidecomplex{IEEE Xtreme and IEEE Activities}{ECE University of Patras}{}{}

  \leftside{}
  \rightsideplain{During the whole duration of my undergraduate studies, I have been regularly involved in the activities of the IEEE community and specifically the IEEEXtreme 24-Hour Programming Competition.}

\end{describesection}


% ********************** %
% *     MEMBERSHIPS    * %
% ********************** %
\opensection{\faGroup}{Memberships}
\begin{describesection}

  \leftside{\bf 2016 - 2019}
  \rightsideplain{IEEE and IEEE Signal Processing Society Student Member}

\end{describesection}

\opensection{\faComments}{Languages}
\begin{describesection}

  \leftside{\color{blue}Greek}
  \rightsideplain{Mother tongue}

  \leftside{\color{blue}English}
  \rightsideplain{Excellent \emph{(C2 Certificate of Proficiency University of Michigan)}}

  \leftside{\color{blue}German}
  \rightsideplain{Good \emph{(B1 Goethe Institute)}}

\end{describesection}

\end{document}
